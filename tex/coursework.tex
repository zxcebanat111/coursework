\documentclass[a4paper,12pt]{extarticle}
\usepackage{geometry}
\usepackage[T1]{fontenc}
\usepackage[utf8]{inputenc}
\usepackage[english,russian]{babel}
\usepackage{amsmath}
\usepackage{amsthm}
\usepackage{amssymb}
\usepackage{fancyhdr}
\usepackage{setspace}
\usepackage{graphicx}
\usepackage{colortbl}
\usepackage{tikz}
\usepackage{pgf}
\usepackage{subcaption}
\usepackage{listings}
\usepackage{indentfirst}
\usepackage[
backend=biber,
style=numeric,
maxbibnames=99
]{biblatex}
\addbibresource{refs.bib}
\usepackage[colorlinks,citecolor=blue,linkcolor=blue,bookmarks=false,hypertexnames=true, urlcolor=blue]{hyperref} 
\usepackage{indentfirst}
\usepackage{mathtools}
\usepackage{booktabs}
\usepackage[flushleft]{threeparttable}
\usepackage{tablefootnote}

\usepackage{chngcntr} % нумерация графиков и таблиц по секциям
\counterwithin{table}{section}
\counterwithin{figure}{section}

\graphicspath{{graphics/}}%путь к рисункам

\makeatletter
% \renewcommand{\@biblabel}[1]{#1.} % Заменяем библиографию с квадратных скобок на точку:
\makeatother

\geometry{left=2.5cm}% левое поле
\geometry{right=1.0cm}% правое поле
\geometry{top=2.0cm}% верхнее поле
\geometry{bottom=2.0cm}% нижнее поле
\setlength{\parindent}{1.25cm}
\renewcommand{\baselinestretch}{1.5} % междустрочный интервал


\newcommand{\bibref}[3]{\hyperlink{#1}{#2 (#3)}} % biblabel, authors, year
\addto\captionsrussian{\def\refname{Список литературы (или источников)}} 

\renewcommand{\theenumi}{\arabic{enumi}}% Меняем везде перечисления на цифра.цифра
\renewcommand{\labelenumi}{\arabic{enumi}}% Меняем везде перечисления на цифра.цифра
\renewcommand{\theenumii}{.\arabic{enumii}}% Меняем везде перечисления на цифра.цифра
\renewcommand{\labelenumii}{\arabic{enumi}.\arabic{enumii}.}% Меняем везде перечисления на цифра.цифра
\renewcommand{\theenumiii}{.\arabic{enumiii}}% Меняем везде перечисления на цифра.цифра
\renewcommand{\labelenumiii}{\arabic{enumi}.\arabic{enumii}.\arabic{enumiii}.}% Меняем везде перечисления на цифра.цифра

\begin{document}
\begin{titlepage}
    \newpage
    
    {\setstretch{1.0}
    \begin{center}
    ПРАВИТЕЛЬСТВО РОССИЙСКОЙ ФЕДЕРАЦИИ\\
    ФГАОУ ВО НАЦИОНАЛЬНЫЙ ИССЛЕДОВАТЕЛЬСКИЙ УНИВЕРСИТЕТ\\
    «ВЫСШАЯ ШКОЛА ЭКОНОМИКИ»
    \\
    \bigskip
    Факультет компьютерных наук\\
    Образовательная программа «Прикладная математика и информатика»
    \end{center}
    }
    
    \vspace{2em}
    УДК 004.421.2 % УДК нужно указывать только для исследовательсвого проекта - удалите эту строку для программного проекта
    \vspace{4em}
    
    \begin{center}
    %Выберите какой у вас проект
    {\bf Отчет об исследовательском проекте на тему:}\\
    %{\bf Отчет о командном исследовательском проекте на тему:}\\
    %{\bf Отчет о программном проекте на тему:}\\
    %{\bf Отчет о командном программном проекте на тему:}\\
    {\bf Гиперэвристика алгоритмов роевого интеллекта}\\
    % строчка ниже нужна только при сдача плана КР, при финальной сдаче закомментируйте ее
    %(промежуточный, этап 1)
    \end{center}
    
    \vspace{2em}
    
    {\bf Выполнил студент: \vspace{2mm}}
    %{\bf Выполнили студенты: \vspace{2mm}}
    
    {\setstretch{1.1}
    \begin{tabular}{l@{\hskip 1.5cm}l}
    группы \#БПМИ236, 2 курса & Коростелев Данил Александрович \\
    %группы \#БПМИ172, 3 курса & Петров Андрей Алексеевич \\
    %группы \#БПМИ173, 3 курса & Иванов Андрей Алексеевич 
    \end{tabular}}
    
    % Обычно у вас есть один научный руководитель, и это человек, с которым вы работаете над проектом. Иногда по формальным причинам у вас будет руководитель (штатный сотрудник Вышки) и соруководитель (тот, с кем вы работаете), — об этом вам сообщит учебный офис (в случае с ВКР) или ЦППРиП (в случае с курсовым проектом). Также, если кто-то дополнительно вам помогал, то его можно указать как консультанта. 
    
    %ваш официальный научник (из ВШЭ)
    \vspace{1em}
    {\bf Принял руководитель проекта: \vspace{2mm}}
    
    {\setstretch{1.1}
    \begin{tabular}{l}
    Родригес Залепинос Рамон Антонио\\
    Научный сотрудник\\
    Факультет компьютерных наук НИУ ВШЭ 
    \end{tabular}}
    
    % со-руководитель (если есть)
    %\vspace{1em}
    %{\bf Соруководитель: \vspace{2mm}}%это ваш официальный научник
    
    %{\setstretch{1.1}
    %\begin{tabular}{l}
    %Петрова Надежда Александровна\\
    %Инженер-исследователь\\
    %ОАО Компания "Нейросети и деревья" 
    %\end{tabular}}
    
    % консультант (если есть)
    %\vspace{1em}
    %{\bf Консультант: \vspace{2mm}}%это ваш официальный научник
    
    %{\setstretch{1.1}
    %\begin{tabular}{l}
    %Иванова Надежда Александровна\\
    %Инженер-исследователь\\
    %ОАО Компания "Нейросети и деревья" 
    %\end{tabular}}
    
    \vspace{\fill}
    
    \begin{center}
    Москва 2025
    \end{center}
    
    \end{titlepage}% это титульный лист - выберите подходящий вам из имеющихся в проекте вариантов (kr - курсовая работа у 3 курса, vkr - выпускная квалификационная работа у 4 курса)
\newpage
\setcounter{page}{2}

{
	\hypersetup{linkcolor=black}
	\tableofcontents
}

\newpage

\newpage
\section*{Аннотация}   % this is how to use russian
Курсовая работа посвящена исследованию и реализации новой гиперэвристики алгоритмов роевого интллекта World Hyper-Heuristic (WHH). В рамках работы проводится анализ существующих алгоритмов роевого интеллекта для дискретных или непрерывных задач оптимизации, таких как алгоритм муравьиной колонии (ACO), оптимизация роем частиц (PSO) и других, с целью выявления их сильных и слабых сторон. Основное внимание уделяется гиперэвристикам — методам, которые позволяют автоматически комбинировать и адаптировать низкоуровневые эвристики для решения сложных оптимизационных задач.

Цель работы заключается в проверке нового алгоритма WHH, на оптимизационных задачах в многомерных пространствах и предложении модификаций, направленных на улучшение качества и скорости сходимости алгоритма. В ходе исследования планируется реализовать предложенный алгоритм и провести его тестирование на стандартных наборах задач оптимизации. Результаты работы будут сравниваться с существующими алгоритмами роевого интеллекта для оценки их производительности и качества решений.

Ключевые аспекты работы включают: изучение теоретических основ роевого интеллекта и гиперэвристик, реализацию архитектуры нового алгоритма, предложение модификаций и оптимизаций, и экспериментальное исследование. Практическая значимость работы заключается в возможности применения разработанного алгоритма для решения задач оптимизации в различных областях, таких как машинное обучение, управление ресурсами, планирование и другие.

\addcontentsline{toc}{section}{Аннотация}

\section*{Ключевые слова}
Обучение с подкреплением, NP-hard, Роевой интеллект
\pagebreak

\section{Введение}

Современные задачи оптимизации, возникающие в различных областях науки и техники, зачастую характеризуются высокой сложностью, многомерностью и наличием ограничений. Традиционные методы оптимизации зачастую оказываются недостаточно эффективными для решения таких задач, что стимулирует развитие новых подходов. Одним из таких направлений являются алгоритмы роевого интеллекта, которые имитируют коллективное поведение природных систем, таких как стаи птиц, колонии муравьев или косяки рыб~\cite{trivedi_varshney}. Эти алгоритмы позволяют с приемлимой скоростью находить субоптимальные решения.

Такие алгоритмы называются метаэвристиками, они начинают с случайного решения и постепенно улучшают его при помощи исследования, иначе говоря, поиска решений, не являющихся близкими к текущему, или эксплуатации, то есть поиска решений среди близких к текущему. Очень важно чтобы алгоритм соблюдал баланс между этими подходами.

Согласно No Free Lunch Theory~\cite{nofreelunch} каждый из этих алгоритмов будет хорош лишь для определенного подмножества проблем и найти универсальный алгоритм не получится. Тут на помощь приходят Гиперэвристики~\cite{burke_etal_2019} - комбинации метаэвристик. Такие алгоритмы переключаются между различными метаэвристиками и позволяют свести задачу к выбору подходящего решения вместо его поиска, а также позволяют лучше контролировать соотношение исследования и эксплуатации.

В 2024 году ученые из университетов Ирана и США предложили свой инновационный алгоритм, основанный на гиперэвристике алгоритмов роевого интеллекта, названный World Hyper-Heuristic~\cite{WHH}. Приведенные в статье данные демонстрируют значительное превосходство WHH над самыми известными метаэвристиками, что не могло не привлечь внимание к теме. В своей курсовой работе я постраюсь воспроизвести результаты полученные в статье и предложу возможные улучшения и оптимизации.
	
\newpage 
\printbibliography[heading=bibintoc] 

% \begin{thebibliography}{0}
% 	\bibitem{chirkova18}\hypertarget{chirkova18}{}
% 	\href{https://arxiv.org/abs/1810.10927}
% 	{Nadezhda Chirkova, Ekaterina Lobacheva, Dmitry Vetrov. Bayesian Compression for Natural Language Processing. In EMNLP 2018.}
% \end{thebibliography}
	
\newpage

\end{document}
